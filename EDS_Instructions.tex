\documentclass[]{article}
\usepackage{lmodern}
\usepackage{amssymb,amsmath}
\usepackage{ifxetex,ifluatex}
\usepackage{fixltx2e} % provides \textsubscript
\ifnum 0\ifxetex 1\fi\ifluatex 1\fi=0 % if pdftex
  \usepackage[T1]{fontenc}
  \usepackage[utf8]{inputenc}
\else % if luatex or xelatex
  \ifxetex
    \usepackage{mathspec}
  \else
    \usepackage{fontspec}
  \fi
  \defaultfontfeatures{Ligatures=TeX,Scale=MatchLowercase}
\fi
% use upquote if available, for straight quotes in verbatim environments
\IfFileExists{upquote.sty}{\usepackage{upquote}}{}
% use microtype if available
\IfFileExists{microtype.sty}{%
\usepackage{microtype}
\UseMicrotypeSet[protrusion]{basicmath} % disable protrusion for tt fonts
}{}
\usepackage[margin=1in]{geometry}
\usepackage{hyperref}
\hypersetup{unicode=true,
            pdftitle={Upload instructions and tips},
            pdfauthor={FES611a},
            pdfborder={0 0 0},
            breaklinks=true}
\urlstyle{same}  % don't use monospace font for urls
\usepackage{color}
\usepackage{fancyvrb}
\newcommand{\VerbBar}{|}
\newcommand{\VERB}{\Verb[commandchars=\\\{\}]}
\DefineVerbatimEnvironment{Highlighting}{Verbatim}{commandchars=\\\{\}}
% Add ',fontsize=\small' for more characters per line
\usepackage{framed}
\definecolor{shadecolor}{RGB}{248,248,248}
\newenvironment{Shaded}{\begin{snugshade}}{\end{snugshade}}
\newcommand{\AlertTok}[1]{\textcolor[rgb]{0.94,0.16,0.16}{#1}}
\newcommand{\AnnotationTok}[1]{\textcolor[rgb]{0.56,0.35,0.01}{\textbf{\textit{#1}}}}
\newcommand{\AttributeTok}[1]{\textcolor[rgb]{0.77,0.63,0.00}{#1}}
\newcommand{\BaseNTok}[1]{\textcolor[rgb]{0.00,0.00,0.81}{#1}}
\newcommand{\BuiltInTok}[1]{#1}
\newcommand{\CharTok}[1]{\textcolor[rgb]{0.31,0.60,0.02}{#1}}
\newcommand{\CommentTok}[1]{\textcolor[rgb]{0.56,0.35,0.01}{\textit{#1}}}
\newcommand{\CommentVarTok}[1]{\textcolor[rgb]{0.56,0.35,0.01}{\textbf{\textit{#1}}}}
\newcommand{\ConstantTok}[1]{\textcolor[rgb]{0.00,0.00,0.00}{#1}}
\newcommand{\ControlFlowTok}[1]{\textcolor[rgb]{0.13,0.29,0.53}{\textbf{#1}}}
\newcommand{\DataTypeTok}[1]{\textcolor[rgb]{0.13,0.29,0.53}{#1}}
\newcommand{\DecValTok}[1]{\textcolor[rgb]{0.00,0.00,0.81}{#1}}
\newcommand{\DocumentationTok}[1]{\textcolor[rgb]{0.56,0.35,0.01}{\textbf{\textit{#1}}}}
\newcommand{\ErrorTok}[1]{\textcolor[rgb]{0.64,0.00,0.00}{\textbf{#1}}}
\newcommand{\ExtensionTok}[1]{#1}
\newcommand{\FloatTok}[1]{\textcolor[rgb]{0.00,0.00,0.81}{#1}}
\newcommand{\FunctionTok}[1]{\textcolor[rgb]{0.00,0.00,0.00}{#1}}
\newcommand{\ImportTok}[1]{#1}
\newcommand{\InformationTok}[1]{\textcolor[rgb]{0.56,0.35,0.01}{\textbf{\textit{#1}}}}
\newcommand{\KeywordTok}[1]{\textcolor[rgb]{0.13,0.29,0.53}{\textbf{#1}}}
\newcommand{\NormalTok}[1]{#1}
\newcommand{\OperatorTok}[1]{\textcolor[rgb]{0.81,0.36,0.00}{\textbf{#1}}}
\newcommand{\OtherTok}[1]{\textcolor[rgb]{0.56,0.35,0.01}{#1}}
\newcommand{\PreprocessorTok}[1]{\textcolor[rgb]{0.56,0.35,0.01}{\textit{#1}}}
\newcommand{\RegionMarkerTok}[1]{#1}
\newcommand{\SpecialCharTok}[1]{\textcolor[rgb]{0.00,0.00,0.00}{#1}}
\newcommand{\SpecialStringTok}[1]{\textcolor[rgb]{0.31,0.60,0.02}{#1}}
\newcommand{\StringTok}[1]{\textcolor[rgb]{0.31,0.60,0.02}{#1}}
\newcommand{\VariableTok}[1]{\textcolor[rgb]{0.00,0.00,0.00}{#1}}
\newcommand{\VerbatimStringTok}[1]{\textcolor[rgb]{0.31,0.60,0.02}{#1}}
\newcommand{\WarningTok}[1]{\textcolor[rgb]{0.56,0.35,0.01}{\textbf{\textit{#1}}}}
\usepackage{graphicx,grffile}
\makeatletter
\def\maxwidth{\ifdim\Gin@nat@width>\linewidth\linewidth\else\Gin@nat@width\fi}
\def\maxheight{\ifdim\Gin@nat@height>\textheight\textheight\else\Gin@nat@height\fi}
\makeatother
% Scale images if necessary, so that they will not overflow the page
% margins by default, and it is still possible to overwrite the defaults
% using explicit options in \includegraphics[width, height, ...]{}
\setkeys{Gin}{width=\maxwidth,height=\maxheight,keepaspectratio}
\IfFileExists{parskip.sty}{%
\usepackage{parskip}
}{% else
\setlength{\parindent}{0pt}
\setlength{\parskip}{6pt plus 2pt minus 1pt}
}
\setlength{\emergencystretch}{3em}  % prevent overfull lines
\providecommand{\tightlist}{%
  \setlength{\itemsep}{0pt}\setlength{\parskip}{0pt}}
\setcounter{secnumdepth}{0}
% Redefines (sub)paragraphs to behave more like sections
\ifx\paragraph\undefined\else
\let\oldparagraph\paragraph
\renewcommand{\paragraph}[1]{\oldparagraph{#1}\mbox{}}
\fi
\ifx\subparagraph\undefined\else
\let\oldsubparagraph\subparagraph
\renewcommand{\subparagraph}[1]{\oldsubparagraph{#1}\mbox{}}
\fi

%%% Use protect on footnotes to avoid problems with footnotes in titles
\let\rmarkdownfootnote\footnote%
\def\footnote{\protect\rmarkdownfootnote}

%%% Change title format to be more compact
\usepackage{titling}

% Create subtitle command for use in maketitle
\providecommand{\subtitle}[1]{
  \posttitle{
    \begin{center}\large#1\end{center}
    }
}

\setlength{\droptitle}{-2em}

  \title{Upload instructions and tips}
    \pretitle{\vspace{\droptitle}\centering\huge}
  \posttitle{\par}
    \author{FES611a}
    \preauthor{\centering\large\emph}
  \postauthor{\par}
      \predate{\centering\large\emph}
  \postdate{\par}
    \date{12/5/2019}


\begin{document}
\maketitle

Hello,

Here's an outline for the final presentation where everyone's work takes
up one slide. There are multiple ways to do this: 1) embed the code and
run it through the R each time 2) do all the preprocessing on your
personal computer -\textgreater{} save final plots as an image
-\textgreater{} upload the image onto a common directory (say on Canvas)
3) repeat steps in (2) but before you upload the plot, make a poster of
it (in powerpoint, inkscape, etc) and then upload that poster (as an
image) onto the directory

Method 1) is going to be hardest to put together, because it'll take a
lot of time to run the code and we'll need all the datafiles (kindly
email \href{akshay.surendra@yale.edu}{Akshay} who'll put it together)
Method 2) is ideal if you don't need too much text around your final
plot - if your plot labels, title etc. are sufficiently informative
Method 3) is ideal if you need to add some description around your final
plot

\#Saving your figures in R

\begin{itemize}
\tightlist
\item
  By default, powerpoint uses a 13.3 inch x 7.5 inch screen size, so
  it'll be great if all your figures are formatted to fit this
\item
  if you're using \texttt{base-R}, you could save plots this way -
\end{itemize}

\begin{Shaded}
\begin{Highlighting}[]
\CommentTok{#all your code goes up here}
\KeywordTok{data}\NormalTok{(}\StringTok{"mtcars"}\NormalTok{)}
\NormalTok{dat<-}\KeywordTok{data.frame}\NormalTok{(}\DataTypeTok{x =} \KeywordTok{scale}\NormalTok{(mtcars}\OperatorTok{$}\NormalTok{mpg),}\DataTypeTok{y =} \KeywordTok{scale}\NormalTok{(mtcars}\OperatorTok{$}\NormalTok{hp))}

\KeywordTok{png}\NormalTok{(}
  \DataTypeTok{file=}\StringTok{"EDSoutput_trial.png"}\NormalTok{,}
  \DataTypeTok{width=}\DecValTok{6}\NormalTok{, }\DataTypeTok{height=}\DecValTok{4}\NormalTok{, }\DataTypeTok{units=}\StringTok{"in"}\NormalTok{,}
  \DataTypeTok{res=}\DecValTok{300}\NormalTok{) }\CommentTok{# this creates an empty .png file with these specification of width, height in inches and with a resolution of 300 pixels/square-inch}

\KeywordTok{plot}\NormalTok{(}\DataTypeTok{x =}\NormalTok{ dat}\OperatorTok{$}\NormalTok{x,}\DataTypeTok{y =}\NormalTok{ dat}\OperatorTok{$}\NormalTok{y) }\CommentTok{#this is the plot that fills the file}

\KeywordTok{dev.off}\NormalTok{() }\CommentTok{#this tells R to stop filling the .png}
\end{Highlighting}
\end{Shaded}

\begin{verbatim}
## pdf 
##   2
\end{verbatim}

\begin{itemize}
\item
  There are lots of resources online to figure out what works best, like
  this
  \href{'https://www.datamentor.io/r-programming/saving-plot/'}{link}.
\item
  Alternatively, you could use \texttt{ggsave()} for a \texttt{ggplot}
  object:
\end{itemize}

\begin{Shaded}
\begin{Highlighting}[]
\KeywordTok{library}\NormalTok{(ggplot2)}
\KeywordTok{data}\NormalTok{(}\StringTok{"diamonds"}\NormalTok{)}
\NormalTok{fig1<-}\KeywordTok{ggplot}\NormalTok{(}\DataTypeTok{data =}\NormalTok{ diamonds) }\OperatorTok{+}\StringTok{ }\KeywordTok{geom_point}\NormalTok{(}\KeywordTok{aes}\NormalTok{(}\DataTypeTok{y=}\NormalTok{carat,}\DataTypeTok{x=}\NormalTok{price)) }\OperatorTok{+}\StringTok{ }\KeywordTok{theme_bw}\NormalTok{() }\CommentTok{#store your figure in an object}

\KeywordTok{ggsave}\NormalTok{(}\DataTypeTok{filename =} \StringTok{"EDoutput_trial.png"}\NormalTok{,}
       \DataTypeTok{plot =}\NormalTok{ fig1,}\DataTypeTok{device =} \StringTok{"png"}\NormalTok{,}
       \DataTypeTok{width =} \DecValTok{5}\NormalTok{,}\DataTypeTok{height =} \DecValTok{6}\NormalTok{,}\DataTypeTok{units =} \StringTok{"in"}\NormalTok{,}\DataTypeTok{dpi =} \DecValTok{300}\NormalTok{) }\CommentTok{#saves the object to a file with a specific format, size and resolution (dpi)}
\end{Highlighting}
\end{Shaded}

\begin{itemize}
\tightlist
\item
  if you have multiple plots, you can use par(mfrow = c(rows,columns))
  in base-R or use grid.arrange(c(plot1,plot2,\ldots plotn),nrow = rows,
  ncol = columns) from the gridExtra package
\item
  Some resources to do this in \href{base-R}{}
\item
  Some resources to do this with ggplot objects with
  \href{\%60gridExtra\%60}{} and \href{\%60cowPlot\%60}{}
\item
  Once you save the plot, you can read images into the .Rpres file
  (example shown in the .Rpres file)
\item
  If you use Python, you an ignore this document and directly upload the
  figure onto the .Rpres file (through R though)
\end{itemize}


\end{document}
